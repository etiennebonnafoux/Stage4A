\subsection{First definitions}

Nous commencerons dans cette section par donner quelques définitions pour introduire la théorie des espaces de Teichmuller.

\begin{dfnt}{Espace de Teichmuller}
Soit $S$ une surface de genre $g$, un marquage de $S$ est un couple $(X,f)$ formé d'une surface de Riemann $X$ et d'un homéomorphisme préservant l'orientaion $f:S \to X$.
Sur l'ensemble des marquages de $S$, nous pouvons faire une relation d'équivalence $(X_1,f_1) \sim (X_2,f_2)$ si il existe $\alpha : X_1 \to X_2 $ tel que $f_2 \circ \alpha \circ f_1^{-1}$ soit un homéomorphisme de $S$ préservant l'orientation et isotope à l'identité.
L'espace des marquages quotienté par la relation s'appelle l'espace de Teichmuller et est noté $\mathcal{T}_g$
\end{dfnt}

\begin{rmq}
Si $g \geq 2$, pour toute courbe simple fermée $\alpha$ de $S$, il existe une unique géodésique fermée de $X$ librement isotope à $f(\alpha)$. Nous noterrons $L_{\alpha}(X)$ sa longeur hyperbolique et nous prenons la topologie la plus faible sur $T_g$ qui rendent ces fonctions continues.
\end{rmq}

\begin{dfnt}{Espace des modules}
On appele groupe modulaire le groupe des homéomorphisme préservant l'oriention de $S$ quotienté par ceux isotope à l'identité.Nous notterons ce groupe $Mod_g$.
Il agit de façon discrète sur $T_g$ et l'espace quotient est appelé espace des modules et est noté $\mathcal{M}_g$
\end{dfnt}

Il est naturel de ce demander à quoi ressemble ces espaces.

\begin{dfnt}{Dehn twist}
Soit $\gamma$ une courbe simple et fermée. Il existe un voisinage tubulaire de $\gamma$ noté $A$ homéomorphe à $[0;1] \times S^{1}$.
On définit le Dehn's twist comme l'homéomorphisme qui vaut l'indentité hors de $A$ et vaut $(t,s) \mapsto (t,e^{2i \pi t} s)$ sur $A$.
\end{dfnt}

\begin{center}
\includegraphics[width=6cm]{Image/Dehn_twist.png}
\end{center}

\begin{rmq}
Le théorème de Lickorisk affirme que le groupe modulaire est engendré par ces Dehn's twist et que plus précisément on peut choisir $2g+1$ générateurs \cite{Lickorish1964AFS}.
\end{rmq}

\begin{dfnt}{Measured foliation}
Given a surface $S$ and a finite set of points $P=(p_1,p_2,...)$, given a open covering $U_i$ on $S-P$, a collection of $C^1$
 real function $\nu_i$ such than $\| d \nu_j \| = \| d \nu_i \|$  on $U_i \cap U_j$,
 and near each singular point $p_s$ a coordinate neighborhood $V$ with complex coordinate $z$ such that $\| d \nu \| = \| Im(z^{\frac{k}{2}}dz) \|$ for some positive integer $k$ called the degree of the singular point,
 leaves of the foliations are the graphs immersed $S$ in along $dv$ is constant. In addition if each boundary circle pf $S$ is contained in a singular leaf, then ti is called a measurd foliation.
\end{dfnt}

The height $h_\gamma(\| d \nu \|)$ of a (free homotopy class) of a loop $\gamma$ on $S$ is the infinimum in the homotopy class of the integral by $\| d \nu \|$
\[
h_\gamma(\| d \nu \|)=inf_{\gamma \equiv \gamma'} \int_gamma \| d \nu \|
\]

The topology on the measured lamination that we will use is the weakest that make the height functions continous.

\begin{rmq}
We won't actually study the set of measured lamination but the equivalence class of \[
h_\gamma(\| d \nu \|)=h_\gamma(\| d \mu \|), \text{ for each loop} \gamma \in S
\].
We can equivalently use Whitehead equivalence relation on singular foliations by collapsing critical intervals to points and taking isotopy of foliation.
\end{rmq}

Let $\mathcal{MF}(S)$ be the space of all equivalence classes of measured foliations.
%TODO dessin et référence
\begin{dfnt}{Lamination}
Une lamination est un ensemble fermé qui est un union (non nécessairement finie) de géodésiques.
Par chaque point x contenu dans $\lambda$ il ne passe que une seul géodésique.
Nous notterons cet espace $\mathcal{ML}(x)$
\end{dfnt}

\begin{dfnt}{Geodesic currents}
Let $\mathcal{M}_\infty$ be the space of unordered pairs of distinct points in $\mathbb{S}^1$ \[
\mathcal{M}_\infty := {(z,w) \in \mathbb{S}^1 \times \mathbb{S}^1 , z \neq w}//(z,w) \equiv (w,z)
\]
Let $G$ be a discret torsion-free group in $PSL(2,\mathbb{R})$ such that $\mathbb{H}//G=S$ is a hyperbolic surface.
A geodesic current $\mu$ on $S$ is a $G$-invariant Radon measure on $\mathcal{M}_\infty$.
We will note $\mathcal{GC}(S)$ the space of geodesic currents.
\end{dfnt}

\begin{rmq}
$\mathcal{GC}(S)$ have a natural topology which is the weak $*$ convergence on continous functions.
\end{rmq}

\begin{rmq}
A mullticurve is a formel sum of geodesics $\gamma = \sum a_i \gamma_i$. The space of lamination is in some aspect the closure of the set of all multicurve
\end{rmq}

\begin{dfnt}{Intersection number}
Consider the square $\mathcal{M}_\infty^2 := \mathcal{M}_\infty \times \mathcal{M}_\infty $. In this space we can consider the open subset $\mathcal{IM}^2_\infty$ corresponding to pair pairs of geodesics which have transversal intersections in $\mathbb{H}$. G act on $\mathcal{IM}^2_\infty$.
If $\mu$ and $\nu$ are geodesic currents in $\mathcal{GC}(S)$, the product $\mu \times \nu$ define a $G$-invariant measure on $\mathcal{IM}^2_\infty$.
Finally if we take the mass of the total space $\mathcal{IM}^2_\infty // G$, the reasult is called the intersection number, $i(\mu,\nu)$
\end{dfnt}

\begin{prop}
\[
i: \mathcal{GC}(S) \times \mathcal{GC}(S) \to \mathbb{R}_+
\]
is continuous and bilinear
\end{prop}

\begin{rmq}
If $\alpha$ and $\beta$ are simple closed geodesics (dirac measure in $\mathcal{GC}(S)$), then the intersection number is the number of intersection between $\alpha$ and $\beta$.
Actually, one can define intersection in this way, first on simple closed geodesic, then by bilinearity on multi-curves and finally by continuity on geodesic current.
\end{rmq}

\begin{rmq}
The topology on $\mathcal{ML}$ is the weakest that make $i(.,.)$ a continous function.
\end{rmq}

\begin{dfnt}{Différentielle quadratique}
Une différentielle quadratique est une section du carré de l'espace tangeant canonique à X. Il s'écrit localement comme $\phi= \phi(z) dz^2$
\end{dfnt}

\begin{rmq}
Si $\phi(p) \neq 0$ on peut trouver une carte contenant $p$ dans laquel $\phi = dz^2$.
Ainsi $\phi$ détermine une métrique plate sur $X$ et un feuilletage $\mathcal{F}$ correspondant aux lignes horizontales.
\end{rmq}

Une différentielle quadratique est dite intégrable si \[
 \| \phi \| = \int_X | \phi | < \infty
\]
Nous notterons $\mathcal{Q}(x)$ l'espace de Banach des différentielles quadratiques intégrables.

\subsection{Flow on Teichmüller space}

We will define the main object of this paper, earthquake flow.

\begin{dfnt}
The earthquake flow is family of maps defined for $t \in \mathbb{R}$
\[
\begin{array}{crcl}

E_t: & \mathcal{ML}\times \mathcal{T}_g & \to & \mathcal{ML}\times \mathcal{T}_g \\

& (\lambda,X) & \mapsto & (\lambda,E_{t\lambda}X)

\end{array}
\]
where $E_\lambda$ is first defined on multi-curves $\gamma =\sum c_i \gamma_i$ by adding $c_i$ to the twist coordinate of $\gamma_i$.
As multi-curves are dense in lamination, we can show that it could be extend to the whole set $\mathcal{ML}$
\end{dfnt}

Thurston show that given two point in the Teichmüller space, there is a lamination $\lambda$ such that the earthquake flow from one point with respect to $\lambda$ reach the other point.

We can ask ourselves what is an invariant measure of this flow.

\begin{dfnt}
The Weil-Peterson form is the the form \[
\omega_{WP} = \sum d l_i \wedge d \tau_i
\]
Where $(l_1,...,\tau_1)$ are the Fenchel-Nielsen according to a pant decomposition.

This give a measure $\mu_{WP}$.
\end{dfnt}

There is a finite measure $\nu_g$ in the Lebesgue measure class on $\mathcal{P}^1 \mathcal{M}_g$ which is invariant under the earthquake flow. This measure projects to the volume form given by $B(X) \times \mu_{WP}$ on $\mathcal{M}_g$, where \[
B(X)=\mu_{Th}(\lambda \in \mathcal{ML}, l_\lambda(X) \leq 1)
\]

%TODO flot horocycle
An other flow is the horocycle flow.

\begin{dfnt}
The horocycle flow is a flow on the bundle of nonzero quadratic differential, $\mathcal{QD}$, of the Teichmuller space given by the unipotent action of \[
u_t=\begin{pmatrix}
1 & t \\
0 & 1
\end{pmatrix}
\]
\end{dfnt}


\subsection{Decomposition of hyperbolic surface}

One way to construct all hyperbolic surface is to decompose them in elementary piece, that we will call pair of pant.

A hyperbolic geometric exercice show that a hexagone which side are geodesics and with right angles is determined by the lenght of three sides which are not consecutifs.
%TODO Faire cette exercice

\begin{center}
\includegraphics[width=12cm]{Image/PairOfPant.jpg}
\end{center}

 On the image above $\gamma_i$, $\gamma_j$ and $\gamma_k$ determined the hexagone. Then we can glued them to have a pair of pant.

 \begin{dfnt}
 A pair of pant is a hyperbolic surface with three geodesic boundaries and no ponctured.
 \end{dfnt}

\begin{rmq}
The pair of pant is uniquely determined by the lenght of the three boundarie geodesics.
\end{rmq}

\begin{rmq}
The lenght of one or more geodesic can go to zero and the boundaries become a ponctured.
\end{rmq}

We can now decompose, with the following theorem, all hyperbolic surfaces in a collection of pair of pants.

\begin{thm}
Let $S$ be a surface of genus $g$ and with $n$ ponctured. There is a set of $3g-3+n$ simple closed curves $(\gamma_1,...,\gamma_{3g-3+n})$ such that $S\ \Cup \gamma_i$ is a disjoint collection of pair of pants.
\end{thm}

\begin{dfnt}
Given a surface $S$ and a pant decomposition $\gamma_1,...,\gamma_{3g-3+n}$, we have a map \[
\mathcal(S) \rightarrow (\mathbb{R^{+}}^{3g-3+n},\mathbb{R}^{3g-3+n}) \\
X \mapsto (l_{\gamma_1}(X),...,l_{\gamma_{3g-3+n}}(X),\tau_{\gamma_1}(X),...,\tau_{\gamma_{3g-3+n}}(X))
\]
This map is injective and is call the Fenchel-Nielsen cordinates.
\end{dfnt}

\subsection{The collaring theorem}

We will give a lemma which is useful to determine if the minimal length of a geodesic crossing an other one.

We define the function \[
\eta(l)= \frac{1}{2} ln(\frac{cosh(l/2)+1}{cosh(l/2)-1})
\]

\begin{dfnt}
Let $\gamma$ be a simple closed geodesic of length $l$ on a hyperbolic surface $X$. If the $\delta$-neighborhood \[
A_\delta(\gamma):= \{ x \in X | d(x,\gamma) < \delta \}
\]
is isometric to the $\delta$-neighbohood of the unique simple closed geodesic on the cylinder of modulus $\frac{\pi}{l}$, we say that $\gamma$ admit a $\delta$-collar
, or that $A_\delta(\gamma)$ is the $\delta$-collar of $\gamma$.
\end{dfnt}

Then we have

\begin{thm}
Let $X$ be a complete hyperbolic surface, and let $\Gamma:={\gamma_1,...}$ be a collection of disjoint simple closed geodesic, each $\gamma_i$ of length $l_i$. Then $A_{\eta(l_i)}(\gamma_i)$ are collars around the $\gamma_i$, and they are disjoint.
\end{thm}

\begin{cor}
Let $X$ be a hyperbolic surface, and $\gamma_1$, $\gamma_2$ two simple closed geodesics on $X$ of lengths $l_1$ and $l_2$. If $l_1 < 2 \eta(l_2)$, then either $\gamma_1=\gamma_2$ or $\gamma \cap \gamma_2 = \emptyset$
\end{cor}

\begin{cor}
Let $X$ be a hyperbolic surface, and let $\gamma_1$, $\gamma_2$ be two simple closed geodesics with lengths $< ln(3+2 \sqrt{2})$. Then either $\gamma_1=\gamma_2$ or $\gamma_1 \cap \gamma_2 = \emptyset$.
\end{cor}

%TODO Mettre des figures et des démonstrations
