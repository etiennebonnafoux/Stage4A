\subsection{Mixing properties of the geodesic and horocyclic flows}

The geodesic flow and the horocyclic flow can act in two situation. In the first one they belong to the action of $SL_2(\mathbb{R})$ on the quotient of the upper half plane by a lattice seen as a finite volume hyperbolic surface. A second manner is in the Teichmuller space and then in the modular space of a given surface.

We will first give the behaviours of two flows we described before.

\begin{thm}
The geodesic flow and the horocycle flow are mixing as seen as flow on surface and on modular space.
\end{thm}
We will just give the proof in the case of a finite volume hyperbolic surface.
The detail of the proof can be found in this course \cite{Mcmullen1998HyperbolicM}.

\begin{proof}
The proof will be in four step, first we will show that the geodesic flow is ergodic, the horocyclic flow is ergodic, the geodesic flow is mixing, finally the horocycle flow is mixing.\\


\emph{Geodesic flow is ergodic}

The following proof is called the Hopf's argument \cite{Hopf1939} and is now common in ergodic theorie.

We will look at the time average of a function $f \in L^2$ which is continuous and compactly support. We will study only the case where the surface is compact, hence $f$ is uniformusly continous. It will be sufficient since this space is dense. We note : \[
F(x) = lim_{T \to \infty} \frac{1}{T} \int_0^T f(g_t x)dt
\]
We want to show that $F$ is almost everywhere constant.


For a point $x$ in the surface, $F(x)$ is constant a.e. on the stable foliation $W^s(x)$ since if two geodesic with the same forward endpoint are asymptotic we can control the difference of the integral with the uniformous continuity of $f$. By the same argument $F$ is constant a.e. on the unstable foliation $W^u(x)$. By Birkhoff's theorem this two constant are the same almost everywhere.

We should now add that the measure is absolutly continous with respect to the stable foliation and the unstable foliation, the definition is given below.
\begin{dfnt}
Let $X$ be a metric space, $ f_t $ of borelian flow whith an invariant borelian measure $\mu$ of finite mass.
 We say that $\mu$ is \emph{absolutly continous with respect to the stable foliation and unstable foliation} if for every $v \in X $ we can find a open neighborhood $U$ of $v$, a $\delta>0$ and a homeomerphism  $\phi:\mathbb{R} \times \mathbb{R} \to U $ such has for every $(x,y) \in \mathbb{R} \times \mathbb{R} $,
\begin{enumerate}
\item $\phi(\{x\}\times \mathbb{R}) \subset W^s(\phi(x,y))$
\item $\phi( \mathbb{R} \times \{y\}) \subset f_{|]-\delta,\delta[} W^u(\phi(x,y))$
\item $\phi^* (\mu_{|U})$ is equivalent to a product measure
\end{enumerate}
\end{dfnt}

We will admit that this is the case in our situation. Moreover we can conclude with the following lemma.

\begin{lem}
Let $(X,\mathcal{T},\mu)$ and $(Y,\mathcal{S},\nu)$ be two probability spaces and $f: X \times Y \to \mathbb{R}$ a function in $L^2$. We suppose that there is $\phi_1:X \to \mathbb{R}$ and $\phi_2:Y \to \mathbb{R}$ two measurable function, such has for $Z \subset X \times Y$ a subset of full $\mu \otimes \nu$ we have :
\[
\forall (x,y) \in Z , f(x,y)=\phi_1 (x) , f(x,y)=\phi_2 (y)
\]
Then $f$ is constant almost everywhere.
\end{lem}

\begin{proof}
According to Fubini's theorem, there is $Y_0 \subset Y$ of full measure and $x_0 \in X$ such has $\{ x_0 \} \times Y_0 \subset Z$. For all $(x,y) \in Z \cap (X \times Y_0)$, the point $(x_0,y)$ is in $Z$ so we have \[
\phi_1(x_0)=f(x_0,y)=\phi_2(y)=f(x,y)
\]
which finish the lemma.
\end{proof}

Putting the pieces together we have that the geodesic flow is ergodic.\\

\emph{Ergodicity of the horocycle flow}

Now if we take $f \in L^2(X)$ a function invariant under the horocycle flow and of mean zero, we want to show that $f=0$ almost everywhere.
Let $G^t$, $H^s$ and $E^r$ correspond to the operators of the different flows. We have the relation \[
H^s=E^r G^t E^{\pi+r}
\]
where $r \to 0$ as $t \to \infty$. Since $H^s f=f$,we have for any $T>0$,\[
f=\frac{1}{T} \int_0^T E^r G^t E^{\pi+r} f dt
\]
Thus for every $g \in L^2(X)$ we have \[
<g,f> = \int_X g \frac{1}{T} \int_0^T E^r G^t E^{\pi+r} f dt
\]
As $r \to 0$ then $t \to \infty$ we can show by controlling the difference that \[
<g,f> = lim_{t \to \infty} \int_X g \frac{1}{T} \int_0^T G^t E^{\pi} f dt
\]
Then as we have shown that the geodesic flow is ergodic, we have by Von Neumann ergodic theorem\[
<g,f> = <g,\int_X E^{\pi} f> =0
\]
This conclude the proof that the horocycle flow is ergodic.\\


\emph{Mixing of the geodesic flow}

We have the relation \[
h_s g_t =g_t h_{s e^{2t}}
\]
Let us take $f_0,f_1 \in C_0(X)$,
we have for small $s$ %TODO rendre ca plus rigoureux
\[
<f_0,g_t f_1 > \simeq <h_{-s} f_0,g_t f_1 >=<f_0 , h_s g_t f_1 >
= <f_0 , g_t h_{s e^{2t}} f_1 > = <g_{-t} f_0 , h_{s e^{2t}}f_1>
\]
Now we have for small $s$, \[
h_{S e^{2t}}f_1 \simeq \frac{1}{S} \int_0^S h_{e^{2t}s} f_1 ds =F_t
\]
and for large t, as the horoclic flow is ergodic \[
F_t = \frac{1}{S} \int_0^S h_{e^{2t}s} f_1 ds \simeq \int_X f_1 = <f_1 , 1 >
\]
So we have \[
<f_0, g_t f_1 > \simeq <g_{-t} f_0, F_t > \simeq <g_{-t} f_0 ,1 ><f_1,1> = <f_0,1><f_1,1>
\]
Which is the mixing of the geodesic flow.\\


\emph{Mixing of the horocyclic flow}

We use again the relation $h^s = e_r g_t e_{\pi + r}$. \[
<h_s f_0 , f_1 > = < g_t  e_{\pi+r} f_0, e_{-r} f_1> \simeq <g_t e_{\pi} f_0, f_1>
\]
for $t$ large (and so $r$ small). By the mixing property of the geodesic flow, this quantity converges to $<f_0,1><f_1,1>$, and the horocyclic flow is mixing.

\end{proof}


Then we have this elementary corollary

\begin{cor}
The Earthquake flow is also mixing with respect to the Liouville measure.
\end{cor}

\begin{proof}
With Mirzakhani's conjugacy, the earthquake flow is conjugated to the horocycle flow which is mixing. This property is carried over.
\end{proof}

\subsection{Rate of mixing of this flows}

Then we want to know at which rate the mixing of the geodesic and horocyclic flow happens. Ratner showed in 1986 \cite{ratner_1987} that the geodesic flow is exponentially mixing and the horocyclic flow has a polynomial rate of mixing. She used representation theory.

We will first give some definitions before getting to the main theorem.

\begin{dfnt}
Let $H$ be a  complex separable Hilbert space, $U(H)$ the group of all unitary transformation of $H$ onto itself, $G$ a Lie group and $T: G \to U(H)$ a unitary representation. We note $T(g)=T_g \in U(H)$.

An element $v \in H$ is called \emph{a $C^k$-vector} for $T$, $k=0,1,...,\infty$ if $g \mapsto T_g(v)$ is a $C^k$-map from $G$ to the Hilbert space. We will note $K(T,p)$ the space of $C^k$-vector
\end{dfnt}

\begin{rmq}
The space of $C^{\infty}$ vector is dense in H.
\end{rmq}

\begin{dfnt}
If $v$ is a $C^1$-vector of $H$ and $X$ is in the Lie algebra of $G$, the \emph{Lie derivative} $L_X v$ is \[
L_X v =lim_{t \to 0} \frac{T(exp tX)v-v}{t}
\]
\end{dfnt}

Now if $\Gamma$ is a Fushian group, $SL_2(\mathbb{R})$ actss on the hyperbolic surface with its measure $(X,\mu)$ as seen as $\mathbb{H}/ \Gamma$.
Then it will also act by a representation $\rho$ on $L^2_0(X,\mu)$ the space of zero average function on the surface.

We want to study the decorrelation induced by some subgroup of $SL_2(\mathbb{R})$. It will link with the following quantity.

\begin{dfnt}
Let $\phi$ and $\psi$ be zero-mean functions in $L^2$, the \emph{matrix coefficient} $C_{\phi,\psi}$ is\[
g \mapsto |<\phi, \rho(g) \psi>|
\]
\end{dfnt}

\begin{rmq}
As the horocyclic flow and the geodesic flow are mixing we have $C_{\phi,\psi}(e_t) \to 0$ and $C_{\phi,\psi}(h_s) \to 0$ for all functions $\phi$ and $\psi$.
\end{rmq}

We can decompose this representation as an integral of irreducible representations \[
L^2_0(X,\mu)=\int H_\zeta d \nu(\zeta)
\]
With $\rho_{\zeta}$ the representation oh $H_{\zeta}$.

This set of representation decompose in three parts. \begin{enumerate}
\item The principal serie
\item The discrete serie
\item The complementary serie
\end{enumerate}

This decomposition is given by the spectrum of the Casimir operator on each $H_\zeta$.
The discrete part of the spectrum is the discrete series, and if $q$ is the eigenvalue, the part $q \geq 1/4$ is the principal series and $q < 1/4$ the complementary series. This decomposition was studied by Bargmann in \cite{10.2307/1969129}.

So we can decompose the measure $\nu=\nu_p + \nu_d + \nu_c$.

Moreover we know that if $\rho_\zeta$ is in the complementary series, there exist $s=s(\zeta) \in ]0;1[$ such that the representation is isomorphic to the representation $\pi_s$ on the Hilbert space \[
\mathcal{H}_s =
\{ f:\mathbb{R} \to \mathbb{C}
, \|f \|^2 = \int_{\mathbb{R} \times \mathbb{R}}
 \frac{f(x) \overline{f(y)}}{|x-y|^{1-s}}dxdy < \infty\}
\]
With the action \[
\pi_s \begin{pmatrix}a & b \\c & d \end{pmatrix}f(x)=\frac{1}{(cx+d)^{1+s}}f(\frac{ax+b}{cx+d})
\]
The representation $\rho$ is isolated from the trivial representation if and only if there exist $\epsilon >0$ such that $s(\zeta)<1-\epsilon$ for $\nu_c$ almost every $\zeta$.


So in our case, the Lie algebra of $SL(2,\mathbb{R})$, is the vector space of $2 \times 2$ matrix with zero trace. A basis for this space is \[
W=\begin{pmatrix} 0 & 1 \\ -1 & 0 \end{pmatrix}, Z=\begin{pmatrix} 1 & 0 \\ 0 & -1 \end{pmatrix}, V=\begin{pmatrix} 0 & 1 \\ 1 & 0 \end{pmatrix}
\]
Which gives the Casimir operator on $C^2$ vectors of $SL(2,\mathbb{R})$, \[
\Omega_T = (L_v^2+L_Z^2-L_W^2)/4
\]

On a irreducible part of $T$, $\Omega_T$ is a scalar multiple of the identity, i.e. \[
\Omega_T v = \lambda v
\]

for some $\lambda=\lambda(T)\in \mathbb{R}$ and all $C^2$-vectors $v \in H(T)$

If $\Gamma$ is a lattice of $SL(2,\mathbb{R})$, $\Omega_T$ create an operator $\Delta$ on $\Gamma \backslash SL(2,\mathbb{R})$. We call $\Lambda(\Delta)$ its spectrum and
\begin{enumerate}
\item $A(\Gamma)=\Lambda(\Gamma) \cup ]-1/4;0[$
\item $B(\Gamma)=sup A(\Gamma)$
\item $C(\Gamma)= -1 + \sqrt{1+4 B(\Gamma)}$
\end{enumerate}

We can now write the theorem about the decay of correlation.

\begin{thm}
Let $\Gamma$ be a lattice in $SL(2,\mathbb{R})$, $M=\Gamma \backslash SL(2,\mathbb{R})$ and $T$ be the regular representation of $SL(2,\mathbb{R})$ on $L^2 (M, \mu)$. Let $v,w\in K(T,p)$, $p>0$, $<w,1>=0$ and $B(t)=<v,w \circ g_t>$, $C(t)=<v,w \circ h_t>$.
Then there exist $t_0=t_0(\Gamma)>0$ such that for all $|t| \geq t_0$ and some $E,F > 0$
\begin{enumerate}
\item $|B(t)| \leq E( b(|t|))^{\alpha(p)}$
\item $|C(t)| \leq F( b(ln|t|))^{\alpha(p)}$
\end{enumerate}
where

\begin{enumerate}
\item $b(t)=e^{\sigma(\Gamma)}$ if $A(\Gamma) \neq \emptyset$
\item $b(t)=e^{-t}$ if $A(\Gamma) = \emptyset$, $sup(\Lambda \cap ]- \infty; -1/4[) < -1/4$ and $-1/4$ is not an eigenvalue of $\Omega$
\item $b(t)=te^{-t}$ if $A(\Gamma) = \emptyset$, $sup(\Lambda \cap ]- \infty; -1/4[)= -1/4$ or $-1/4$ is an eigenvalue of $\Omega$
\end{enumerate}
and $\alpha(p)$ is
\begin{enumerate}
\item $1$ if $p \geq 3$
\item $\frac{2p}{2p+1}$ if $2 \leq p < 3$
\item $\frac{2p}{2p+3}$ if $1 \leq p < 2$
\item $\frac{p}{p+3}$ if $0<p<1$
\end{enumerate}
\end{thm}

\begin{rmq}
We are only interested in functions in $L^2(X,\mu)$ which depend only on the footprint of the tangent space and are constant on the tangent space. This implies that they are all in $K(T,\infty)$.
\end{rmq}

On the other we can make the path backward, by learning information on the representation via the mixing rate of the flow it generates. This way is taken in the appendix B of \cite{2005math.....11614A}.

\begin{dfnt}
Let $G$ be a locally compact $\sigma$-compact group. A continuous unitary
 representation of $G$ is said to have \emph{almost invariant vectors}
 if for every $\epsilon > 0$ and for every compact subset $K \subset G$, there exists a unit vector $V$ such that $\|g*v-v\| < \epsilon$ for all $g \in K$.

 A unitary action which does not have almost invariant vectors is said to be \emph{isolated from trivial representation}.

 If $G$ is a semi-simple Lie group, a representation which is isolated from trivial representation is alos said to have a spectral gap.
\end{dfnt}

With this definition, we can write the following theorem.

\begin{prop}
Let consider a representation of $SL(2,\mathbb{R})$ by a measure preserving action of automorphisms of a probability space. Let $\rho$ be the representation associate on $H$ the space of the $L^2$ function with zero averages. Assume there is $\delta \in ]0;1[$ and a dense subset of the subspace of $SO(2,\mathbb{R})$-invariant function $H' \subset H$ consisting of functions $\phi$ for which the correlations $<\phi, \rho(g_t) \time \phi>$, $g_t=\begin{pmatrix} e^t & 0 \\ 0 & e^{-t} \end{pmatrix} $
, decays like $O(e^{- \delta t})$. Then $\rho$ is isolated from the trivial representation.

\end{prop}
