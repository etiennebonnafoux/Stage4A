In this master thesis, I wanted to show how the hyperbolicity provides us with powerful tools. These tools, among which the collaring lemma or the Fenchel-Nielsen coordinates can be cited, are a cornerstone of understanding flows on the underlying surface. The objects introduced (lamination, foliation, quadratic differential) are diverse yet all linked and interconnected. This strong consistency is key to transporting qualitative properties as we did in the case of the ergodicity between the horocyclic flow and the earthquake flow. But some quantitative properties are still unknow and require more efforts. Although we can compute decay rates of geodesic and horocyclic flows, we can not directly find this rate for the earthquake flow. So far, we're still working on deriving a bound in the simplest possible case, which is the once punctured torus.


In the future, efforts will be set on finishing the derivations and computations of the mixing rate of the earthquake flow on the once punctured torus. Then we hope that this simple case will yield enlightning result and allow us to apply similar arguments in higher genus. This will provide us with an upper bound of the rate of mixing. However, we expect that showing that the bounds obtained are tight and examining in which situations they can be reached will require us to use other approaches and tools.

I am very enthousiastic about these perspectives and look forward to working on them in my Ph.D under the supervision of Carlos Matheus.

I would like to thanks Carlos Matheus who shared his knowledge and guided me on this topic. I am also grateful to Ser-Wei Fu, who helped us a lot to understand the bound we can give to the length of systole along the earthquake flow. My thanks also go to Ming Kun Liu who helped me getting a hand on some tricky points, and to Athénaïs Gautier who corrected some of my mistakes.

%Bisous tu es une copine géniale
