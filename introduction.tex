
\hspace{20 px} Since Bernhard Riemann, mathematician knew that only a finite number of parameter can describe a geometric surface. Considering this, we can be interested in the set of all geometry we can give to a given surface, modulo composition by map isotopic the identity, this set is called the Teichmuller space. Moreover other problem rise shortly after this definition. How can we deform in the natural way a geometry of a surface into an other? What does it mean that two geometries are closed one to each other? What are the natural boundary of the Teichmuller space?

\begin{wrapfigure}{r}{5cm}
  \centering
  \includegraphics[width=4cm]{Image/Teichmuller.jpeg}
  \caption{The mathematician Oswald Teichmuller}
\end{wrapfigure}

\vspace{10 px}

Oswald Teichmuller, a german mathematician, gave answers to this question in the year preceding the second World war.He create the first metric on this space by finding a solution to an extremal problem: between two hyperbolic geometry on the same surface is there a function which minimize the deformation. the answer is not onmy yes, but this fonction is unique. It naturraly create a distance in the now called Teichmuller space by considering the logarithm of the deformation of the extremal function.


\vspace{10 px}

Thurston then add other important step to this theory. He underligne the role of lamination which are a generalisation of simple closed curve. And he create the earthquake flow which turn to play an important role in Teichmuller theory.
Kerckhoff used this tool to show the Nielsen realisation conjecture in 1983 \ref{NielsenRealizationPro} which state that every finite subgroup of the mapping class group have a fixed point in the Teichmuller space.

\vspace{10 px}

An important question in Hyperbolic geometry is the asymptotic number of closed geodesic. To begin we can ask the number $\pi(X,L)$ of geodesic on a hyperbolic surface of length less than $L$. The answer was found by Delsarte, Huber and Selberg and is called the prime number theorem for hyperbolic surfaces (because of the ressemblance to the prime number theorem). It states that \[
\pi(X,L) \equiv e^{L} / L
\]
as $L \to \infty$.
A much harder question was to find the number, $\sigma(X,L)$, of simple (which don't intersect themselves) closed geodesic of length less than $L$ on a hyperbolic surface $X$. It was find years later, in Mirzakhani's PhD, and we have \[
\sigma(X,L) \equiv C_{X}L^{6g-6}
\]
As $L \to \infty$ where $g$ is the genus of the surface $X$ and $C_{X}$ is a constant which depent of the geometry $X$.
To do that Myriam Mirzhakhani conjugate the earthquake flow to the horocycle flow. This step give that the Earthquake flow is ergodic and allow us to use Birkoff theorem to understand asymptotic quantity.

The question is now to give error term to this quantity and to do that we need to understand better the earthquake flow.

In this master thesis, I will first give an introduction to Teichmuller theory and some useful tool for . Then we will review the proof of the Mirzhakani conhjugacie between the horocyclic flow and the earthquake flow. Finally we will discuss of a special case which can be the most simple example of hyperbolic surface, that is the once ponctured torus.
