
%%-----------------------------
%%      the top matter
%%-----------------------------
\title{Figures}

\begin{figure}[h]
\centering
\begin{tikzpicture}[scale=1,cap=round]
\tkzInit[xmin=-4, xmax=4, ymin=-4, ymax=4]
\tkzClip
%Draw circle
\tkzDefPoint(0,0){C} %define center
\tkzDefPoint(0,4){N} %define "north"
\tkzDefPoint(0,3){NC}
\tkzDefPoint(0,-4){S} %define "south"
\tkzDefPoint(0,-2){SC}
\tkzDrawCircle[thick](C,N){circleC} %circle centered in C passing through N, named circleC
\draw[thick] (S) -- (N); %Lines between S and N

%%% North semi-arc

\tkzDefPointWith[orthogonal](NC,C) \tkzGetPoint{ortC_NC} %One point that is on the orthogonal of NC-C (passing by NC) called ortC_NC

\tkzInterLC(NC,ortC_NC)(C,N) \tkzGetPoints{N1}{N2} %The intersection between the line NC, ortC_NC and the circle centered in C passing by N, the two points are called N1 and N2

\tkzDefPointWith[orthogonal](N1,C) \tkzGetPoint{ortC_N1} %One point that is on the orthogonal of N1-C (passing by N1) called ortC_N1
\tkzDefPointWith[orthogonal](N2,C) \tkzGetPoint{ortC_N2} %One point that is on the orthogonal of N2-C (passing by N2) called ortC_N2

\tkzInterLL(N1,ortC_N1)(N2,ortC_N2) \tkzGetPoint{CN} %The intersection between the line N1-ortC_N1 and the  line N1-ortC_N1, the points is called CN

\begin{scope}
\tkzInit[xmin=0, xmax=4]
\tkzClip
\tkzClipCircle(C,N) %only what is inside the circle centered in C passing through N
\tkzDrawCircle[thick](CN,N1){circleN} %circle centered in CN passing through N1, named circleN
\end{scope}
%\tkzMarkRightAngle[size=.2](C,N2,CN)


%%% South semi-arc

\tkzDefPointWith[orthogonal](SC,C) \tkzGetPoint{ortC_SC} %One point that is on the orthogonal of SC-C (passing by SC) called ortC_SC


\tkzInterLC(SC,ortC_SC)(C,S) \tkzGetPoints{S1}{S2} %The intersection between the line SC, ortC_SC and the circle centered in C passing by S, the two points are called S1 and S2


\tkzDefPointWith[orthogonal](S1,C) \tkzGetPoint{ortC_S1} %One point that is on the orthogonal of S1-C (passing by S1) called ortC_S1
\tkzDefPointWith[orthogonal](S2,C) \tkzGetPoint{ortC_S2} %One point that is on the orthogonal of S2-C (passing by N2) called ortC_S2

\tkzInterLL(S1,ortC_S1)(S2,ortC_S2) \tkzGetPoint{CS} %The intersection between the line S1-ortC_S1 and the  line S1-ortC_S1, the points is called CS

\begin{scope}
\tkzInit[xmin=0, xmax=4]
\tkzClip
\tkzClipCircle(C,S) %only what is inside the circle centered in C passing through S
\tkzDrawCircle[thick](CS,S1){circleS} %circle centered in CS passing through S1, named circleS
\end{scope}
%\tkzMarkRightAngle[size=.2](C,S1,CS)


%%% East semi arc
\tkzInterLL(S1,ortC_S1)(N2,ortC_N2) \tkzGetPoint{CE} %The intersection between the line S1-ortC_S1 and the  line S1-ortC_S1, the points is called CS
\begin{scope}
\tkzClipCircle(C,S) %only what is inside the circle centered in C passing through S
\tkzDrawCircle[thick,green](CE,S1){circleE} %circle centered in CN passing through N1, named circleE
\end{scope}
\tkzMarkRightAngle[size=.2](C,N2,CE)
\tkzMarkRightAngle[size=.2](C,S1,CE)


\tkzDefLine[orthogonal=through CE](N,S)\tkzGetPoint{X}
\tkzInterLC(CE,X)(CE,S1) \tkzGetFirstPoint{M}
\tkzInterLL(M,X)(C,N) \tkzGetPoint{CM} %The intersection between the line SC, ortC_SC and the circle centered in C passing by S, the two points are called S1 and S2

\draw[>=triangle 45, <->, thick] (CM) -- node[below] {$\eta(\ell)$} (M);

\tkzInterLC(N,S)(CN,N1) \tkzGetSecondPoint{NM}
\tkzInterLC(N,S)(CS,S1) \tkzGetFirstPoint{SM}
\draw[>=triangle 45, <->, red, thick] (NM) -- node[left] {$\ell$} (SM);

% Mark right angles with vertical
\tkzDefPointWith[orthogonal](SM,SC)\tkzGetPoint{SME}
\tkzMarkRightAngle[size=.2](CS,SM,SME)
\tkzDefPointWith[orthogonal,K=-1](NM,NC)\tkzGetPoint{NME}
\tkzMarkRightAngle[size=.2](CN,NM,NME)

%\tkzDrawPoints(C, N, S, S1, N1, S2, N2, X, M, CM)
%\tkzLabelPoints[below](C, N, S, S1, N1, S2, N2, X, M, CM)
\end{tikzpicture}
\caption{In red the segment of length $l$ and in green the geodesic whose endpoint are orthogonal projection of the end of the segment.}
\end{figure}
