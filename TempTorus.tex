
\begin{thm} Let $\nu(S_{1,1})$ be the finite measure on $\mathcal{P}^1 \mathcal{M}(S_{1,1})$. Then \[
\nu(S_{1,1})\{ (X,\lambda) \in \mathcal{P}^1 \mathcal{M}(S_{1,1}) | l_{sys}(X,\lambda) < \epsilon \} = O(\frac{\epsilon}{log \epsilon})
\] as $\epsilon \rightarrow 0 $
\end{thm}

%Portion de preuve
\hrulefill

Let fix $\epsilon=l_{sys}(X)$, and $T > 0 $, then $\exists N=N(\epsilon,T)$ such that $\forall n \geq N$ \[
| l_{sys}E_t(X,\lambda)-l_{sys}E_t(X,\frac{\gamma_n}{l_{\gamma_n}(X)})| < \epsilon
\]

So we calculate $T_n$ such that $\forall |t| < T_n$,
 $l_{sys}E_t(X,\frac{\gamma_n}{l_{\gamma_n}(X)}) < 2 \epsilon$
and $T=liminf T_n$. If $T>0$, $\exists N_1$ such that $\forall n \geq N_1$, $T_n \geq \frac{T}{2}$.

Then we set $N_2=N(\epsilon,T/2)$ and we have $l_{sys}E_t(x,\lambda) \leq 2.5 \epsilon $, $\forall 0 \leq t \leq T/2$


\hrulefill

We have two useful tools to understand the length fonction along a earthquake path.

\begin{lem}
Let $X \in \mathcal{T}(S_g)$, and $\gamma$ a curve which is part of a pant decomposition.
 $\chi_s$
  is the twist of length $s$ around $\gamma$, and $b$ a closed curve with $i(b,\gamma) > 0$ then $s \mapsto l_{\chi_s}(b)$ is strictly convex.
\end{lem}
\cite{farb2011primer} proposition 10.8

\begin{lem}
If $\alpha$ is a closed curve, $\gamma$ an other closed curve and $\lambda$ a lamination.\[
\begin{array}{crcl}

\frac{dl_{\alpha}}{dt}(0) & = & \sum_{p_i \in \alpha \cap \gamma} cos(\theta_{p_i}) \\

\frac{dl_{\alpha}}{dt}(0) & = & \int_{ \alpha} cos(\theta) d \theta

\end{array}
\]
\end{lem}
\cite{NielsenRealizationPro} Corollary 3.3 and 3.4
